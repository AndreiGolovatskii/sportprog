\begin{problem}{В поисках невест}{brides.in}{brides.out}{2 секунды}{64 мегабайта}

Однажды король Флатландии решил отправить $k$ своих сыновей на поиски невест.
Всем известно, что во Флатландии $n$ городов, некоторые из которых соединены 
дорогами. Король живет в столице, которая имеет номер $1$, а город с номером 
$n$ знаменит своими невестами.

Итак, король повелел, чтобы каждый из его сыновей добрался по дорогам из города
$1$ в город $n$. Поскольку, несмотря на обилие невест в городе $n$, красивых среди них
не так много, сыновья опасаются друг друга. Поэтому они хотят добраться до цели
таким образом, чтобы никакие два сына не проходили по одной и той же дороге (даже
в разное время). Так как король любит своих сыновей, он хочет, чтобы среднее время
сына в пути до города назначения было минимально.

\InputFile

В первой строке входного файла находятся числа $n$, $m$ и $k$ --- количество городов и
дорог во Флатландии и сыновей короля, соответственно ($2 \le n \le 200$, $1 \le m \le 2000$,
$1 \le k \le 100$). Следующие $m$ строк содержат по три целых положительных числа каждая 
--- города, которые соединяет соответствующая дорога и время, которое требуется
для ее прохождения (время не превышает $10^6$). По дороге можно перемещаться в любом 
из двух направлений, два города могут быть соединены несколькими дорогами.


\OutputFile

Если выполнить повеление короля невозможно, выведите на первой строке число
$-1$. В противном случае выведите на первой строке минимальное возможное среднее
время (с точностью 5 знаков после десятичной точки), которое требуется сыновьям,
чтобы добраться до города назначения, не менее чем с пятью знаками после 
десятичной точки. В следующих $k$ строках выведите пути сыновей, сначала число дорог в
пути, и затем номера дорог в пути в том порядке, в котором их следует проходить.
Дороги нумеруются, начиная с единицы, в том порядке, в котором они заданы во
входном файле.

\Example

\begin{example}
\exmp{
5 8 2
1 2 1
1 3 1
1 4 3
2 5 5
2 3 1
3 5 1
3 4 1
5 4 1
}{
3.00000
3 1 5 6
3 2 7 8
}%
\end{example}

\end{problem}
