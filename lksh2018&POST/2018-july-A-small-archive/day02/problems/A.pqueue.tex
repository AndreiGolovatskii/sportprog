\begin{problem}{Персистентная очередь}{pqueue.in}{pqueue.out}{2 секунды}{64 мегабайта}

Реализуйте персистентную очередь.

\InputFile

Первая строка содержит количество действий $n$ ($1 \le n \le 200\,000$).

В строке номер $i + 1$ содержится описание действия $i$:

\begin{shortitems}
\item \texttt{1 t m} --- добавить в конец очереди номер $t$ ($0 \le t < i$) число $m$;
\item \texttt{-1 t} --- удалить из очереди номер $t$ ($0 \le t < i$) первый элемент.
\end{shortitems}

В результате действия $i$, описанного в строке $i + 1$ создается очередь номер $i$.

Изначально имеется пустая очередь с номером ноль.

Все числа во входном файле целые, и помещаются в знаковый 32-битный тип.


\OutputFile
Для каждой операции удаления выведите удаленный элемент на отдельной строке.

\Examples

\begin{example}
\exmp{10
1 0 1
1 1 2
1 2 3
1 2 4
-1 3
-1 5
-1 6
-1 4
-1 8
-1 9
}{1
2
3
1
2
4
}%
\end{example}

\end{problem}
