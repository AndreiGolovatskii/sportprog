% Problem author: Sergey Kopeliovich
% Source: SIS.July.2014

\begin{problem}{ВнутренНЯя точка 2}{inside2.in}{inside2.out}{2 секунды}{64 мегабайта}

Дан совсем невыпуклый \emph{простой} $N$-угольник и $K$ точек.
Напомним, $N$-угольник называется простым, если не имеет ни самопересечений, ни самокасаний.
Для каждой точки нужно определить, где она находится "--- внутри, на границе, или снаружи.

\InputFile

В первой строке дано целое число $T$ "--- количество тестов. 

Далее идут $T$ тестов.
Тесты разделены переводом строки. 

$N$ ($3 \le N \le 10^5$). Далее $N$ точек "--- вершины многоугольника.

$K$ ($0 \le K \le 10^5$). Далее $K$ точек "--- запросы.

Все координаты "--- целые числа по модулю не превосходящие $10^9$.


Суммарное количество $N$ и $K$ не превосходит $2 \cdot 10^6$.
\OutputFile

Для каждого запроса одна строка "--- \texttt{INSIDE}, \texttt{BORDER} или \texttt{OUTSIDE}.

Тесты следует разделять переводом строки. 

\Examples

\begin{example}
\exmp{
1
4
0 0
2 0
2 2
0 2
4
1 1
0 0
0 1
0 3
}{
INSIDE
BORDER
BORDER
OUTSIDE
}%
\end{example}

\end{problem}
