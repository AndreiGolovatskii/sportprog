\begin{problem}{Intercity Express}{intercity.in}{intercity.out}{5 секунд}{256 мегабайт}

%Author: Andrew Stankevich, ASC 39

Андрей разрабатывает систему для продажи железнодорожных билетов.
Он собирается протестировать ее на Междугородней Экспресс линии, которая соединяет два больших города и имеет $n-2$ промежуточных станций, то есть в итоге есть $n$ станций, пронумерованных от 1 до $n$.

В Междугороднем Экспресс поезде есть $s$ мест, пронумерованных с 1 до $s$.
В тестирующем режиме система имеет доступ к базе данных, содержащей проданные билеты в направлении от станции 1 до станции $n$ и должна отвечать на вопросы, можно ли продать билет от станции $a$ до станции $b$, и если да, нужно найти минимальный номер места, которое свободно на протяжении всего пути между $a$ и $b$. 

Изначально система имеет только доступ на чтение, то есть даже если есть свободное место, она должна сообщить об этом, но не должна изменять данные.

Помогите Андрею протестировать его систему написанием программы, которые будет находить ответы на вопросы.

\InputFile
Первая строка содержит число $n$~--- количество станций, $s$~--- количество мест и $m$~--- количество уже проданных билетов ($2 \le n \le 10^9$,
$1 \le s \le 100\,000$, $0 \le m \le 100\,000$).

В следующих $m$ строках описаны билеты, описание каждого билета состоит из трех чисел: $c_i$, $a_i$ и $b_i$~--- номер места, которое занимает владелец билета, номер станции, с которой продан билет и номер станции, до которой продан билет ($1 \le c_i \le s$, $1 \le a_i < b_i \le n$).

Следующие строки содержат число $q$~--- количество запросов ($1 \le q \le 100\,000$). 
Специальное значение $p$ должно поддерживаться в течение считывания запросов. Изначально $p = 0$.

Следующие $2q$ строк описывают запросы. Каждый запрос описывается двумя числами: $x_i$ и $y_i$ ($x _i \le y_i$).

Чтобы получить города $a$ и $b$ между которыми нужно проверить наличие места, используется следующая формула:

$a = x_i + p$, $b = y_i + p$. Ответ на запрос~--- число 0, если нет места на каждом отрезке между $a$ и $b$, или минимальный номер свободного места.

После ответа на запрос, надо приравнять число $p$ полученному ответу на запрос.

\OutputFile
Для каждого запроса выведите ответ на него.

\Examples

\begin{example}
\exmp{5 3 5
1 2 5
2 1 2
2 4 5
3 2 3
3 3 4
10
1 2
1 2
1 2
2 3
-2 0
2 4
1 3
1 4
2 5
1 5
}{1
2
2
3
0
2
0
0
0
0
}%
\end{example}

\Note

Обратите внимание, что запросы выглядят так:  

$(1, 2)$, $(2, 3)$, $(3, 4)$, $(4, 5)$, $(1, 3)$, $(2, 4)$, $(3, 5)$, $(1, 4)$, $(2, 5)$, $(1, 5)$.


\end{problem}
