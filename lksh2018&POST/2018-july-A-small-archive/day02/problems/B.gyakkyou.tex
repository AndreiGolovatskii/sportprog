\begin{problem}{k-я статистика на отрезке}
{gyakkyou.in}{gyakkyou.out}
{4 seconds}{256 mebibytes}

\noindent{}$k$-ю статистику на отрезке $[l..r]$ массива $A$ можно определить следующим способом:

\vspace*{-0.5em}
\begin{verbatim}
int get( int l, int r, int k ) {
  B = отрезок [l..r] массива A
  sort(B)
  return B[k]
}
\end{verbatim}
\vspace*{-0.4em}
Дан массив. Ваша задача "--- много раз отвечать на запрос ``$k$-я статистика на отрезке''.

\InputFile

Первая строка содержит целое число $N$, количество чисел в массиве ($1 \le N \le 450\,000$).

Вторая строка используется, чтобы сгенерировать массив $a_1, a_2, \dots, a_N$.

Она содержит три целых числа $a_1$, $l$ и $m$ ($0 \le a_1, l, m < 10^9$).
\vspace*{-0.2em}
$$a_i = (a_{i-1} \cdot l + m) \bmod 10^9, \quad 2 \le i \le N$$
\vspace*{-1.1em}

Третья строка содержит целое число $B$~--- число групп запросов ($1 \le B \le 1000$).
$B$ следующих строк описывают группы запросов.
Каждая группа описывается десятью целыми числами.
Первым идет число $G$~--- количество запросов.
Затем следуют $x_1$, $l_x$ и $m_x$, потом $y_1$, $l_y$ и $m_y$, и наконец, $k_1$, $l_k$ и $m_k$ 
($1 \le x_1 \le y_1 \le N$, $1 \le k_1 \le y_1 - x_1 + 1$, $0 \le l_x, m_x, l_y, m_y, l_k, m_k < 10^9$).
Они используются, чтобы сгенерировать вспомогательную последовательность $x_g$ и $y_g$ и текущие параметры $i_g$, $j_g$ и $k_g$ для $1 \le g \le G$:
\vspace*{-0.2em}
$$
\begin{array}{ccll}
x_g & = & ((i_{g - 1} - 1) \cdot l_x + m_x) \bmod N) + 1, & 2 \le g \le G \\
y_g & = & ((j_{g - 1} - 1) \cdot l_y + m_y) \bmod N) + 1, & 2 \le g \le G \\
i_g & = & \min(x_g, y_g), & 1 \le g \le G \\
j_g & = & \max(x_g, y_g), & 1 \le g \le G \\
k_g & = & (((k_{g - 1} - 1) \cdot l_k + m_k) \bmod (j_g - i_g + 1)) + 1,
& 2 \le g \le G \\
\end{array}
$$
\vspace*{-0.7em}

Сгенерированные параметры означают, что в $g$-м запросе,
Нужно узнать $k_g$-ую статистику на отрезке $[i_g, j_g]$ массива.
Общее количество запросов по всем группам не превышает $600\,000$.

Формат столь необычный, чтобы тесты были маленькие по объёму.

\OutputFile

Выведите одно число: сумму всех полученных статистик.

\Example

\begin{example}
\exmp{
5
1 1 1
5
1 1 0 0 3 0 0 2 0 0
1 2 0 0 5 0 0 3 0 0
1 1 0 0 5 0 0 5 0 0
1 3 0 0 3 0 0 1 0 0
1 1 0 0 4 0 0 1 0 0
}{
15
}%
\end{example}

\medskip

\Note

Будьте аккуратны при генерации запросов. Часто ошибаются именно в этой части.

% \Hint

% Задача имеет простое решение персистентным деревом отрезков за $\O(\log n)$ на запрос.

% Прочитать краткое описание решение можно \href{http://acm.math.spbu.ru/~sk1/mm/lections/zksh2015-ds/mipt-2014-burunduk1-ds.pdf}{здесь} (стр.8).

\end{problem}
