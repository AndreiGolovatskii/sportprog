\begin{problem}{Под-бор}{trie.in}{trie.out}{2 секунды}{256 мебибайт}

%Author: Sergey Kopeliovich
%Text Author: Petr Mitrichev

\emph{Бором} называется подвешенное дерево, на каждом из рёбер которого
написано по символу, причём символы, написанные на рёбрах,
выходящих из общей вершины-родителя, различны.
Будем называть направление от родителя к детям ``вниз''. 
Назовем \emph{вхождением строки $s$ в бор} такую вершину бора, от которой
можно пройти несколько шагов вниз таким образом, что встретившиеся символы
образуют строку $s$.

Даны бор и несколько строк, найдите сумму количеств вхождений этих строк
в этот бор.

\InputFile

В первой строке входного файла записано единственное число $n$, $1 \le n \le 100\,000$~---
количество вершин бора.
В следующих $n$ строках описаны вершины бора. В ($i+1$)-й строке описаны дети $i$-й вершины:
число $k_i$ ее детей, затем $k_i$ пар из номера вершины-ребёнка и символа,
написанного на соответствующем ребре. Номер родителя всегда меньше номера
ребёнка; корнем бора является вершина номер 1.

В ($n+2$)-й строке записано количество $m$ ($1 \le m \le 100\,000$) строк для поиска.
В следующих $m$ строках перечислены сами строки. Входные строки непусты, а их суммарная
длина не превышает $100\,000$ символов.

Все символы, написанные на рёбрах, а также все символы, составляющие строки --- маленькие
латинские буквы.

\OutputFile

Выведите одно число --- сумму количеств вхождений.

\Example
\begin{example}
\exmp{
7
2 2 a 4 b
2 3 a 6 b
0
1 5 b
1 7 b
0
0
4
b
bb
bbb
bb
}{
9
}%
\end{example}

\end{problem}
