\begin{problem}{Поиск набора образцов}{console.in}{console.out}{1 секунда}{64 мегабайта}{Z2}

Напишите программу, которая для каждой строки из заданного набора $S$
проверяет, верно ли, что она содержит как подстроку одну из строк из
набора $T$.

\InputFile

Первая строка входного файла содержит натуральное число $n$ --- 
количество строк в наборе $T$ ($n \le 1000$).
Каждая из следующих $n$ строк содержит непустую строку длины не более 80-ти
символов.

Оставшаяся часть файла содержит строки из набора $S$.
Каждая строка состоит из \texttt{ASCII} символов с кодами от 32 до 126
включительно. Строка может быть пустой и ее длина не превышает
250-ти символов.

Гарантируется, что размер входного файла не превышает 1 Мбайт.

\OutputFile

В выходной файл выведите все строки из набора~$S$ (в том порядке, в котором они находятся во
входном файле), содержащие как подстроку
хотя бы одну строку из набора~$T$.

\Example

\begin{example}
\exmp{
3
gr
sud
abc
lksh
sudislavl
kostroma
summer
group a'
}{
sudislavl
group a'
}%
\end{example}

\Note

Если у вас \t{WA 36}, вы неправильно читаете входные данные. Строки могут состоять только из пробелов.

\end{problem}
