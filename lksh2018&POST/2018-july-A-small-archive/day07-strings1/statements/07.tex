\documentclass[12pt,a4paper,oneside]{article}

\usepackage{cmap}
\usepackage[T2A]{fontenc}
\usepackage[utf8]{inputenc}
\usepackage[english,russian]{babel}
\usepackage[russian]{olymp}
\usepackage{graphicx}
\usepackage{amsmath,amssymb}
\usepackage{epigraph}
\usepackage[russian,colorlinks=true,urlcolor=red,linkcolor=blue]{hyperref}
\usepackage{enumerate}
\usepackage{datetime}
\usepackage{color}
\usepackage{lastpage}
\usepackage{import}
\usepackage{verbatim}

\def\ID{7}
\def\NAME{День \NO{\ID}, Строки-1}%
\def\WHERE{ЛКШ.2018.Июль, Берендеевы Поляны}%
\def\DATE{12 июля 2018}%

\renewcommand{\t}{\texttt}
\renewcommand{\le}{\leqslant}
\renewcommand{\ge}{\geqslant}

\binoppenalty=10000
\relpenalty=10000
\exhyphenpenalty=10000

\newcommand{\ProblemLabel}{undefined}
\newcommand{\ProblemTL}{undefined}
\newcommand{\ProblemML}{undefined}
\newcommand{\ProblemName}{undefined}

\newcommand{\q}[1]{\langle #1 \rangle}
\newcommand\NO[1]{\t{\##1}}
\def\O{\mathcal{O}}
\def\EPS{\varepsilon}
\def\SO{\Rightarrow}
\def\EQ{\Leftrightarrow}
\def\t{\texttt}
\def\XOR{\text{ {\raisebox{-2pt}{\ensuremath{\Hat{}}}} }}
\def\LINE{\vspace*{-1em}\noindent \underline{\hbox to 1\textwidth{{ } \hfil{ } \hfil{ } }}}

\def\probl#1#2#3#4{
  \renewcommand{\ProblemName}{#1}
  \renewcommand{\ProblemLabel}{#2}
  \renewcommand{\ProblemTL}{#3}
  \renewcommand{\ProblemML}{#4}
  \input ../problems/#2.#1.tex
}
          
\newcommand{\Section}[2]{
  \hbox{\hspace{1em}}
  \vspace*{-2.5em}
  \section*{\color{#1}{#2}}
  \addcontentsline{toc}{section}{\color{#1}{#2}}
  \vspace*{-0.5em}
}

\def\myindent{\hspace*{\parindent}\unskip}

\contest
{\NAME}%
{\WHERE}%
{\DATE}%

%\sectionfont{\fontsize{8}{8}\selectfont}

\definecolor{dkgreen}{rgb}{0,0.6,0}
\definecolor{brown}{rgb}{0.5,0.5,0}

\def\compact{
  \setlength{\parskip}{0pt}
  \setlength{\itemsep}{0pt}
}

\begin{document}

\vspace*{-2.8em}

\tableofcontents

\vspace*{0.8em}
\LINE
\vspace*{0.8em}

\noindent{}В некоторых задачах большой ввод и вывод. 
Пользуйтесь \href{https://ejudge.lksh.ru/A/lib/example_io.cpp.html}{быстрым вводом-выводом}.

\vspace{0.6em}
\noindent{}В некоторых задачах нужен STL, который активно использует динамическую память (set-ы, map-ы)
\href{https://ejudge.lksh.ru/A/lib/example_mem.cpp.html}{переопределение стандартного аллокатора} ускорит вашу программу.

\pagebreak

\Section{red}{Задачи}

\probl{dictionary}{A}{0.3 sec}{256 mb}    % 0.187
\probl{console}{B}{0.15 sec}{256 mb}      % 0.073, 0.025
\probl{shifts}{C}{0.2 sec}{256 mb}        % 0.095, 0.069
\probl{towers}{D}{0.1 sec}{256 mb}        % 0.030
\probl{toptzsuffarray}{E}{3 sec}{256 mb}  % 0.492

\Section{red}{Гробы}

\probl{trie}{F}{0.2 sec}{256 mb}          % 0.047

\end{document}
