% Source: PTZ Winter 2016, Moscow SU Trinity contest

\begin{problem}{MeX на пути дерева}{\textsl{standard input}}{\textsl{standard output}}{6 seconds}{256 mebibytes}

Вам дано дерево из $n$ вершин. Каждому ребру дерева сопоставлено неотрицательное целое число $x_i$. 
Ваша задача -- ответить на $q$ запросов. $j$-й из запросов -- найти
минимальное неотрицательное целое число $y$, отсутствующее на пути между вершинами $a_j$ и $b_j$.

\InputFile

Первая строка содержит числа $n$ и $q$ ($2 \leq n \leq 10^5$, $1 \leq q \leq 10^5$), количество вершин в дереве и число запросов.

Следующие $n - 1$ строка содержит описания рёбер дерева, тройки целых чисел $u_i$, $v_i$, $x_i$ ($1 \leq u_i, v_i \leq n$, $u_i \neq v_i$, $0 \leq x_i \leq  10^9$).

Следующие $q$ строк содержат пары целых чисел $a_j$, $b_j$ ($1 \leq a_j, b_j \leq n$), обозначающие запросы.

\OutputFile

На каждый запрос $a_j$, $b_j$ на отдельной строке выведите минимальное целое неотрицательное $y$, не лежащее на пути между $a_j$ и $b_j$.

\Example

\begin{example}
\exmp{7 6
2 1 1
3 1 2
1 4 0
4 5 1
5 6 3
5 7 4
1 3
4 1
2 4
2 5
3 5
3 7
}{0
1
2
2
3
3
}%
\end{example}

\end{problem}
