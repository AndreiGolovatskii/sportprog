% Problem author: Denis Denisov
% Text author: Denis Denisov
% Tests author: Sergei Rogulenko

\begin{problem}{Вас снова замяукали!}
{labyrinth.in}{labyrinth.in}
{2 секунды}{256 мебибайт}

Два котёнка попали в запутанный лабиринт со множеством комнат и переходов
между ними. Котята долго по нему плутали, обошли все комнаты по много раз,
нашли выход (да даже и не один, а несколько), в общем, изучили там всё, что
смогли. Теперь этот лабиринт котята используют в своих играх.

Чаще всего котята играют в следующую игру: начиная в какой-то комнате
лабиринта, котята поочерёдно выбирают, в какую из комнат им перейти.
Котята изначально находятся в одной комнате и ходят вместе.
Как только котёнок, который должен выбрать следующую комнату, не может
этого сделать, он признаётся проигравшим. Обычно в таких играх выигрывающий
игрок стремится выиграть как можно быстрее, а проигрывающий стремится как
можно дольше оттянуть свое поражение. Но у котят свои представления о
победе и поражении. Если котёнок знает, что, начиная из текущей комнаты,
он выиграет (вне зависимости от действий другого котёнка), то он стремится
играть как можно дольше, чтобы продлить себе удовольствие от выигрыша
(естественно, при этом выигрывающий котёнок должен гарантировать себе, что
будет постоянно уверен в выигрыше).
Котёнок, который знает, что проиграет (при условии, конечно, что другой
котёнок будет действовать оптимально), старается проиграть как можно
быстрее, чтобы начать новую игру, в которой и взять реванш.

Если котята будут ходить бесконечно долго, но никто из них не сможет выиграть,
то котята считают игру завершившейся вничью и замяукивают Вас.

Вас попросили для каждой комнаты в лабиринте узнать, выиграет или проиграет
котёнок, начинающий ходить из данной комнаты. Если котёнок, начинающий из
этой комнаты, выигрывает, требуется узнать максимальное количество ходов,
которое он сможет играть, если же проигрывает "--- минимальное количество,
которое ему придётся играть.

\InputFile

В первой строке ввода находятся два числа $n$ и $m$ "--- число комнат и
переходов между комнатами в лабиринте ($1 \le n \le 100\,000$,
$0 \le m \le 100\,000$). Далее следует $m$ строк с описаниями переходов.
Описание перехода состоит из
двух чисел $a$ и $b$, означающих, что котёнок, начинающий игру в комнате
с номером $a$, может выбрать комнату $b$ в качестве следующей.

\OutputFile

Выведите $n$ строк "--- для каждой комнаты результат игры для котёнка, 
который начнет игру из этой комнаты. Если игра закончится вничью,
выведите <<\t{DRAW}>>. Если начинающий котёнок выиграет, выведите
<<\t{WIN K}>>, где \t{K} "--- количество ходов, которые сможет 
играть выигрывающий котёнок. Если котёнок сможет играть сколь угодно долго,
сохраняя возможность в любой момент выиграть, выведите <<\t{WIN INF}>>.
Если котёнок, начинающий из этой комнаты, проиграет, выведите
<<\t{LOSE K}>>, где \t{K} "--- количество ходов, которые
придется играть проигрывающему котёнку. Если же котёнку придется играть
сколь угодно долго, при том, что его соперник сможет в любой момент выиграть,
выведите <<\t{LOSE INF}>>.

\Examples

\begin{example}
\exmp{
4 4
1 2
1 3
2 4
3 4
}{
LOSE 2
WIN 1
WIN 1
LOSE 0
}%
\exmp{
6 6
1 2 
2 3
3 4
4 1
4 5
5 6
}{
DRAW
DRAW
DRAW
DRAW
WIN 1
LOSE 0
}%
\exmp{
6 6
1 2
2 3
3 4
4 1
2 6
4 5
}{
LOSE INF
WIN INF
LOSE INF
WIN INF
LOSE 0
LOSE 0
}%
\end{example}

\end{problem}
