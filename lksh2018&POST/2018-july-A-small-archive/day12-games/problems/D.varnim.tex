% Problem author: Ivan Kazmenko, Vitaliy Valtman
% Text author: Ivan Kazmenko
% Tests author: Ivan Kazmenko

\begin{problem}{Вариация Нима}
{varnim.in}{varnim.out}
{2 секунды}{256 мегабайт}

На столе лежат $n$ кучек камней: $a_1$ камней в первой кучке,
$a_2$ камней во второй, \ldots, $a_n$ в $n$-ой. Двое играют
в игру, делая ходы по очереди. За один ход игрок может либо
взять произвольное ненулевое количество камней (возможно, все) из одной
любой кучки, либо произвольным образом разделить любую существующую кучку,
в которой не меньше двух камней, на две непустые кучки. Проигрывает
тот, кто не может сделать ход. Кто выигрывает при правильной игре?

\InputFile

В первой строке задано целое число $t$~--- количество
тестов ($1 \le t \le 100$).
Следующие $t$ строк содержат сами тесты. Каждая из них начинается
с целого числа $n$~--- количества кучек ($1 \le n \le 100$).
Далее следует $n$ целых чисел $a_1$, $a_2$, $\ldots$, $a_n$
через пробел~--- количество камней в кучках ($1 \le a_i \le 10^9$).

\OutputFile

Выведите $t$ строк; в $i$-ой строке выведите ``\t{FIRST}'',
если в $i$-ом тесте при правильной игре выигрывает первый игрок,
и ``\t{SECOND}'', если второй.

\Example

\begin{example}
\exmp{
3
1 1
2 1 1
3 1 2 3
}{
FIRST
SECOND
FIRST
}%
\end{example}

\end{problem}
