% Автор идеи: Сергей Копелиович
% Источник: пары СПбАУ, весна 2017/18

\begin{problem}{Альфа Дерево}
{ab.in}{ab.out}
{1 секунда}{256 мегабайт}{}

У вас есть полное бинарное дерево глубины $n$ ($0 \le n \le 30$).

В дереве $2^n$ листьев, они пронумерованы слева направо числами от $0$ до $2^n{-}1$.

В $i$-м листе записано число $x_i = (a i^2 + b i + c) \bmod m$. % Линейная рекуррента. Любая функция, которую можно, но не слишком тривиально быстро посчитать .

Есть фишка, которая изначально находится в корне дерева. Двое играют в игру, двигая фишку вниз по дереву.
Когда фишка достигает листа дерева, игра заканчивается. Цель первого игрока -- максимизировать число в листе, цель второго -- минимизировать.

\InputFile

Числа $n$, $a$, $b$, $c$, $m$. При этом $10 \le m \le 10^9$.

Все $a$, $b$, $c$ сгенерированы равномерным распределением на $[0, m)$.

\OutputFile

Выведите результат игры при оптимальной игре обоих.

\Examples

\begin{example}
\exmp{
3 10 7 9 20
}{
11
}%
\end{example}

\Note

Взятие остатка по модулю -- небыстрая операция. Чем их меньше, тем лучше.

\end{problem}
