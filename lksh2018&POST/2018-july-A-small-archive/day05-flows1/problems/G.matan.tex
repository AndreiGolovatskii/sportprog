% Source: SummerSchool Camp 2007-07-03

% Used: SIS.2009.July.A0

\begin{problem}{Матан}{matan.in}{matan.out}{1 секунда}{64 мегабайта}

В Университете города М. проводят эксперимент. Преподаватели сами решают, что они будут читать в рамках того или иного курса.
И вот преподаватель математического анализа (в простонародье "--- матана) оценил по некоторым критериям все известные ему темы в данном курсе.
В результате этой ревизии каждой теме сопоставлено некоторое целое число (возможно, отрицательное) "--- полезность данной темы.
Профессор хочет максимизировать суммарную полезность прочитанных им тем, но не все так просто. Для того что бы студенты поняли некоторые темы, необходимо, 
чтобы были прочитаны так же некоторые другие темы, так как некоторые доказательства базируются на фактах из других тем.
Однако если существует цикл из зависимостей тем, то их все можно прочитать, и на качестве понимания материала студентами
это не скажется.

Вас попросили составить список 
тем, которые профессор должен прочитать, таким образом, чтобы студенты все поняли, и 
суммарная полезность курса была максимальна. 

\InputFile

Первая строка входного файла содержит одно число "--- $N$ ($1 \le N \le 200$).
Вторая строка содержит $N$ целых чисел, не превосходящих по модулю $1\,000$ "--- полезности каждой темы.
Далее следуют $N$ строк с описанием зависимостей тем.
Каждое описание начинается количеством тем, которые необходимо понять для понимания данной темы. Потом следуют номера этих тем, разделенные пробелами.
Суммарное количество рёбер не более $1\,800$.

\OutputFile

Выведите единственное число "--- максимально возможную суммарную полезность прочитанного материала.

\Examples

\begin{example}%
\exmp{
4
-1 1 -2 2
0
1 1
2 4 2
1 1
}{
2
}%
\exmp{
3
2 -1 -2
2 2 3
0
0
}{
0
}%
\end{example}

\end{problem}
