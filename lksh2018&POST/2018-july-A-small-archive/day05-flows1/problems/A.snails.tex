% Problem author: Sergey Kopeliovich
% Text author: Mikhail Belous
% Tests author: Mikhail Belous

\begin{problem}{Улиточки}{snails.in}{snails.out}
{2 секунды}{256 мегабайт}

Две улиточки Маша и Петя сейчас находятся в на лужайке с абрикосами и хотят добраться до своего домика.
Лужайки пронумерованы числами от 1 до $n$ и соединены дорожками (может быть несколько дорожек соединяющих две лужайки,
могут быть дорожки, соединяющие лужайку с собой же).
В виду соображений гигиены, если по дорожке проползла улиточка, то вторая по той же дорожке уже ползти не может.
Помогите Пете и Маше добраться до домика.

\InputFile

В первой строке файла записаны четыре целых числа --- $n$, $m$, $s$ и $t$
(количество лужаек, количество дорог, номер лужайки с абрикосами и номер домика).
В следующих $m$ строках записаны пары чисел. Пара чисел $(x, y)$ означает,
что есть дорожка с лужайки $x$ до лужайки $y$ (из-за особенностей улиток и местности дорожки односторонние).
Ограничения:  $2 \le n \le 10^5, 0 \le m \le 10^5, s \not= t$.

\OutputFile

Если существует решение, то выведите \t{YES} и
на двух отдельных строчках сначала последовательность лужаек
для Машеньки (дам нужно пропускать вперед),
затем путь для Пети.
Если решения не существует, выведите \t{NO}. Если решений несколько, выведите любое.

\Example

\begin{example}
\exmp{
3 3 1 3
1 2
1 3
2 3
}{
YES
1 3 
1 2 3 
}%
\end{example}

\Note

Дан орграф, найти два непересекающихся по ребрам пути из $s$ в $t$, вывести вершины найденных путей.

\end{problem}
