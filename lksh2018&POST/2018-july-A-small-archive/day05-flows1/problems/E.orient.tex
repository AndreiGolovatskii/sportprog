%
% Зимние всероссийские школьные учебно-тренировочные сборы по информатике
% понедельник, 2 декабря 2013 года
%
% Условие: Сергей Копелиович
% Тесты: Сергей Копелиович
% Решения: Сергей Копелиович
% Идея: Сергей Копелиович
%

\begin{problem}{Ориентируй меня полностью!}{}{}
{2 секунды}{256 мебибайт}{}

Вам дан неориентированный граф без петель и кратных рёбер.
Ваша задача "--- ориентировать граф таким образом, чтобы
максимальная исходящая степень была бы минимально возможной.
                                                                                    
\InputFile

В первой строке заданы числа $n$ и $m$ "--- количество вершин и рёбер в графе
($1 \le n \le 25\,000; 0 \le m \le 25\,000$).
В следующих $m$ строках даны пары чисел от $1$ до $n$ "--- рёбра графа.

\OutputFile

Выведите минимально возможную максимальную степень.
Далее выведите $m$ целых чисел от $0$ до $1$.
Если $i$-е ребро было задано парой чисел $a$, $b$, то ноль означает,
что оно после ориентации ведёт из $a$ в $b$, а единица "--- что из $b$ в $a$.

\Examples

\begin{example}
\exmp{%
4 4
1 2
1 3
4 2
4 3
}{%
1
0 1 1 0
}%
\exmp{%
5 5
1 2
2 3
3 1
1 4
1 5
}{%
1
0 0 0 1 1
}%
\end{example}

\end{problem}
