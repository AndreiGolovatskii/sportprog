\documentclass[12pt,a4paper,oneside]{article}

\usepackage{cmap}
\usepackage[T2A]{fontenc}
\usepackage[utf8]{inputenc}
\usepackage[english,russian]{babel}
\usepackage[russian]{olymp}
\usepackage{graphicx}
\usepackage{amsmath,amssymb}
\usepackage{epigraph}
\usepackage[russian,colorlinks=true,urlcolor=red,linkcolor=blue]{hyperref}
\usepackage{enumerate}
\usepackage{datetime}
\usepackage{color}
\usepackage{lastpage}
\usepackage{import}
\usepackage{verbatim}

\def\ID{1}
\def\NAME{День \NO{\ID}, LCA и RMQ}%
\def\WHERE{ЛКШ.2018.Июль, Берендеевы Поляны}%
\def\DATE{5 июля 2018}%

\renewcommand{\t}{\texttt}
\renewcommand{\le}{\leqslant}
\renewcommand{\ge}{\geqslant}

\binoppenalty=10000
\relpenalty=10000
\exhyphenpenalty=10000

\newcommand{\ProblemLabel}{undefined}
\newcommand{\ProblemTL}{undefined}
\newcommand{\ProblemML}{undefined}
\newcommand{\ProblemName}{undefined}

\newcommand{\q}[1]{\langle #1 \rangle}
\newcommand\NO[1]{\t{\##1}}
\def\O{\mathcal{O}}
\def\EPS{\varepsilon}
\def\SO{\Rightarrow}
\def\EQ{\Leftrightarrow}
\def\t{\texttt}
\def\XOR{\text{ {\raisebox{-2pt}{\ensuremath{\Hat{}}}} }}
\def\LINE{\vspace*{-1em}\noindent \underline{\hbox to 1\textwidth{{ } \hfil{ } \hfil{ } }}}

\def\probl#1#2#3#4{
  \renewcommand{\ProblemName}{#1}
  \renewcommand{\ProblemLabel}{#2}
  \renewcommand{\ProblemTL}{#3}
  \renewcommand{\ProblemML}{#4}
  \input ../problems/#2.#1.tex
}
          
\newcommand{\Section}[2]{
  \hbox{\hspace{1em}}
  \vspace*{-2.5em}
  \section*{\color{#1}{#2}}
  \addcontentsline{toc}{section}{\color{#1}{#2}}
  \vspace*{-0.5em}
}

\def\myindent{\hspace*{\parindent}\unskip}

\contest
{\NAME}%
{\WHERE}%
{\DATE}%

%\sectionfont{\fontsize{8}{8}\selectfont}

\definecolor{dkgreen}{rgb}{0,0.6,0}
\definecolor{brown}{rgb}{0.5,0.5,0}

\def\compact{
  \setlength{\parskip}{0pt}
  \setlength{\itemsep}{0pt}
}

\begin{document}

\vspace*{-2.8em}

\tableofcontents

\vspace*{0.8em}
\LINE
\vspace*{0.8em}

%\pagebreak

% \noindent{}Вы не умеете читать/выводить данные, открывать файлы? Воспользуйтесь 
% \href{http://acm.math.spbu.ru/~sk1/algo/sum/}{примерами}.
% \vspace{0.6em}

%\url{http://acm.math.spbu.ru/~sk1/algo/input-output/io_export.cpp.html} \\
\noindent{}В некоторых задачах большой ввод и вывод. 
Пользуйтесь \href{https://ejudge.lksh.ru/A/lib/example_io.cpp.html}{быстрым вводом-выводом}.

% \vspace{0.6em}
% \noindent{}В некоторых задачах нужен STL, который активно использует динамическую память (set-ы, map-ы)
% \href{http://acm.math.spbu.ru/~sk1/algo/memory.cpp.html}{переопределение стандартного аллокатора} ускорит вашу программу.

% \vspace{0.6em}
% \noindent{}Обратите внимание на компилятор \t{GNU C++11 5.1.0 (TDM-GCC-64) inc},
% который позволяет пользоваться \href{http://acm.math.spbu.ru/~sk1/algo/lib/optimization.h.html}{дополнительной библиотекой}.
% Под ним можно сдать \href{http://acm.math.spbu.ru/~sk1/algo/lib/}{вот это}. 

\pagebreak

%\Section{red}{Must have}

\probl{sum}{A}{0.05 sec}{256 mb}
\probl{lca_rmq}{B}{2.5 sec}{256 mb}
\probl{union}{C}{0.75 sec}{256 mb}
\probl{roads}{D}{0.6 sec}{256 mb}
\probl{tree}{E}{1.5 sec}{256 mb}

%\Section{dkgreen}{Обязательные задачи}
%\Section{black}{Дополнительные задачи}

\end{document}
