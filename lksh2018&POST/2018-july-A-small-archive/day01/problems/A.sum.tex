\begin{problem}{Сумма}{sum.in}{sum.out}{2 секунды}{64 мегабайта}{E}

Дан массив из $N$ элементов, нужно научиться находить сумму чисел на отрезке.

\InputFile

Первая строка содержит два целых числа $N$ и $K$ --- число чисел в массиве и количество запросов.
$(1 \leq N \leq 100 \, 000)$, $(0 \leq K \leq 100 \, 000)$. Следующие $K$ строк содержат запросы

\begin{shortitems}
\item{``\t{A i x}'' --- присвоить $i$-му элементу массива значение $x$ ($1 \leq i \leq n$, $0 \leq x \leq 10^9$)}
\item{``\t{Q l r}'' --- найти сумму чисел в массиве на позициях от $l$ до $r$. ($1 \leq l \leq r \leq n$)}
\end{shortitems}

Изначально в массиве живут нули.

\OutputFile

На каждый запрос вида Q l r нужно вывести единственное число --- сумму на отрезке.

\Examples

\begin{example}
\exmp{
5 9
A 2 2
A 3 1
A 4 2
Q 1 1
Q 2 2
Q 3 3
Q 4 4
Q 5 5
Q 1 5
}{
0
2
1
2
0
5
}%
\end{example}

\Note

Обыкновенное дерево отрезков.

\end{problem}
