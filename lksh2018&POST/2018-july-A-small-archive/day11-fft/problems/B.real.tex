\begin{problem}{Раздвоение}{real.in}{real.out}{2 секунды}{256 мегабайт}

% Author: Zhukov Dmitry

Обозначим две последовательности действительных чисел $x(k)$ и $y(k)$.
Определим последовательность комплексных чисел $z(k)$: $z(k) = x(k) + iy(k)$.

Пусть $FFT_N(k, z) = \sum\limits^{N-1}_{n=0}z_ne^{2 \pi i k n / N}$.

Аналогичным образом определяются $FFT_N(k, x)$ и $FFT_N(k, y)$.

По вычисленным значениям $FFT_N(k, z)$
восстановите значения $FFT_N(k, x)$ и $FFT_N(k, y)$.

\InputFile

В первой строке входного файла записано целое число $N$ ($1 \le N \le 2^{30}$,
$N$ является степенью двойки).
Далее следуют целые неотрицательные числа $A$, $B$, $C$, $D$, $E$, $F$, не превосходящие 1000.
Для экономии времени ввода значения $FFT_N(k, z)$ нужно будет вычислять по следующим формулам:

$FFT_N(k, z).real = ((A + B \cdot k)\ xor \ (C \cdot k)) \cdot 10^{-3}$,

$FFT_N(k, z).imag = ((D + E \cdot k)\ xor \ (F \cdot k)) \cdot 10^{-3}$,

где $FFT_N(k, z).real$ и $FFT_N(k, z).imag$ --- действительная и мнимая части соответственно.

Затем дано число $M$ --- количество запросов ($1 \le M \le 10^5$).
Далее следуют $M$ целых чисел $q_j$ ($0 \le q_j < N$).
%Далее следуют значения $FFT_N(k, z)$ для $k = 0, \ldots, N-1$ ---
%пары действительных чисел, не превосходящих 1000 по абсолютному значению и
%заданных не более, чем с 6-ю знаками после запятой,
%--- действительная и мнимая части соответствующего числа.

\OutputFile

В выходной файл выведите $M$ строк.
В $j$-ой строке --- значения $FFT_N(q_j, x)$ и $FFT_N(q_j, y)$.
Значения должны отличаться от правильных не более, чем на $10^{-4}$.

\Examples

\begin{example}
\exmp{
2
1000 0 0 0 0 0
2
0 1
}{
1.0 0.0 0.0 0.0
1.0 0.0 0.0 0.0
}%
\exmp{
4
0 100 300 500 100 200
4
0 1 2 3
}{
0.000 0.000 0.500 0.000
0.504 0.140 0.516 0.176
0.656 0.000 0.812 0.000
0.504 -0.140 0.516 -0.176
}%
\exmp{
1048576
999 998 997 996 995 994
3
17 239239 2011
}{
540.737 -1587.741 1589.778 539.689
2404.809 531.421 1359.578 1569.751
3678.277 -523.243 526.382 3664.887
}%
\end{example}

\end{problem}
