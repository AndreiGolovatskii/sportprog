\begin{problem}{АВЛ-деревья}{avl.in}{avl.out}{5 секунд}{256 мегабайт}
%\begin{problem}{AVL Trees}{avl.in}{avl.out}{5 seconds}{256 megabytes}

% Author: Andrew Stankevich

АВЛ-дерево --- сбалансированное по высоте двоичное дерево поиска:
для каждой его вершины высота её двух поддеревьев различается не более чем на 1.
АВЛ-деревья названы по первым буквам фамилий их изобретателей,
Г. М. Адельсона-Вельского и Е. М. Ландиса.
%AVL trees invented by Russian scientists Adelson-Velskiy and Landis are used
%for \emph{sorted collection} data structure. The rooted binary tree is 
%called \emph{balanced} if for each vertex the height of its left subtree
%and the height of its right subtree differ by at most one. The balanced
%binary search tree is called the AVL tree.

Для фиксированного количества вершин может существовать несколько АВЛ-деревьев.
Например, существует шесть АВЛ-деревьев, состоящих из пяти вершин.
%There can be several AVL trees with the given number of vertices. For example,
%there are 6 AVL trees with 5 vertices, they are shown on the picture below.

\begin{center}
\includegraphics{pics/avl.1}\hspace{0.5cm}
\includegraphics{pics/avl.2}\hspace{0.5cm}
\includegraphics{pics/avl.3}\hspace{0.5cm}
\includegraphics{pics/avl.4}\hspace{0.5cm}
\includegraphics{pics/avl.5}\hspace{0.5cm}
\includegraphics{pics/avl.6}
\end{center}

Также деревья с одинаковым количеством вершин могут иметь различную высоту.
Например, существуют деревья из семи вершин с высотами 2 и 3 соответственно.
%Also the tree with the given number of vertices can have different height,
%the picture below shows AVL trees with 7 vertices that have height 2 and 3,
%respectively. 

\begin{center}
\includegraphics{pics/avl.7}\hspace{0.5cm}
\includegraphics{pics/avl.8}
\end{center}

Требуется по заданным $n$ и $h$ найти количество АВЛ-деревьев, состоящих из
$n$ вершин и имеющих высоту $h$. Так как ответ может быть очень большим,
требуется найти остаток от деления искомого количества на 786433.
%Given $n$ and $h$ find the number of AVL trees that have $n$ vertices and
%height $h$. Since the answer can be quite large, return the answer
%modulo $786\,433$.

\InputFile

Во входном файле даны числа $n$ и $h$ ($1 \le n \le 65535$,
%Input file contains $n$ and $h$ ($1 \le n \le 65\,535$,
$0 \le h \le 15$).

\OutputFile

Выведите одно число --- остаток от деления количества АВЛ-деревьев,
состоящих из $n$ вершин и имеющих высоту $h$, на 786433.
%Output one number --- the number of AVL trees with $n$ vertices that have 
%height $h$, modulo $786\,433$. 

\Example

\begin{example}
\exmp{
7 3
}{
16
}%
\end{example}
\bigskip

\Note

786433 --- простое число, $786433 = 3 \cdot 2^{18} + 1$.
%Note that $786\,433$ is prime, and $786\,433 = 3 \cdot 2^{18} + 1$.

\end{problem}

