% Автор идеи: Сергей Копелиович
% Источник: практика для студентов СПб АУ 2016/17

\begin{problem}{Деление многочленов}
{divpoly.in}{divpoly.out}
{1 секунда}{256 мегабайт}{}

Даны два многочлена с коэффициентами из $\mathbb{Z}/7\mathbb{Z}$.

Старший коэффициент обоих {\bf не равен нулю}.

Нужно поделить их с остатком.

\InputFile

Каждая из двух строк задаёт описание многочлена.
Многочлен $a_kx^k + \dots + a_2x^2 + a_1x + a_0$ описывается числом $k$ ($0 \le k \le 50\,000$) и $k+1$ числами от $0$ до $6$:
$a_k, \dots, a_2, a_1, a_0$.

\OutputFile

На первой строке многочлен-частное. 
На второй строке многочлен-остаток.
Выводите многочлены в том же формате. 
Если многочлен -- тождественный ноль, для него $k = 0$.

\Examples

\begin{example}
\exmp{
3 1 1 1 1
1 1 1
}{
2 1 0 1
0 0
}%
\exmp{
3 1 1 3 1
2 1 1 1
}{
1 1 0 
1 2 1 
}%
\exmp{
8 2 1 2 1 2 1 2 1 2
4 1 2 3 4 5
}{
4 2 4 2 5 2 
3 3 1 3 6 
}%
\end{example}

\end{problem}
