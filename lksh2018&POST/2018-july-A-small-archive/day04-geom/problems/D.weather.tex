% Источник: timus.1894 (Не лётная погода)

\begin{problem}{Не лётная погода}
{weather.in}{weather.out}
{1 секунда}{256 мегабайт}{}

Все вылеты из аэропорта Кольцово отложены. Главный диспетчер заявил, что, пока над территорией аэропорта висит подозрительная грозовая туча, он не позволит ни одному борту подняться в воздух. Впрочем, он заверил пассажиров, что ветер сносит тучу со скоростью один метр в секунду и скоро небо над аэропортом очистится. Правда, о направлении ветра он ничего не сказал.
Один из пассажиров нашёл в интернете спутниковый снимок территории аэропорта, сделанный ровно минуту назад. По снимку можно точно определить положение тучи на тот момент времени. Точные координаты аэропорта тоже найти нетрудно. Хватит ли этой информации, чтобы вычислить минимальное время, через которое диспетчер может дать добро на вылет самолётов?

\InputFile

Тучу и территорию аэропорта на снимке можно приближённо считать невырожденными строго выпуклыми многоугольниками на плоскости. 
В первой строке записаны целые числа $n$ и $m$ -- количество вершин в многоугольнике, задающем территорию аэропорта, и количество вершин в многоугольнике, задающем тучу ($3 \le n, m \le 50\,000$). В следующих $n$ строках записаны координаты территории аэропорта в порядке обхода против часовой стрелки. 
Далее в аналогичном формате задано положение тучи. Все координаты указаны в метрах и являются целыми числами, не превосходящими $10^8$ по модулю. 
Гарантируется, что на снимке туча закрывает хотя бы одну точку внутренней области аэропорта.

\OutputFile

Выведите единственное число -- минимальное количество секунд, которое может пройти до того момента, когда ни одна точка территории аэропорта не будет находиться под тучей. Выведите ответ с абсолютной или относительной погрешностью не более $10^{-6}$. Если уже сейчас туча может не закрывать ни одну точку территории аэропорта, выведите $0$.

\Examples

\begin{example}
\exmp{
4 4
400 400
600 400
600 600
400 600
0 0
1000 0
1000 1000
0 1000
}{
540.0
}%
\end{example}

\end{problem}
