% Источник: timus.1271 (Лоцман)

\begin{problem}{Лоцман}
{pilot.in}{pilot.out}
{1 секунда}{256 мегабайт}{}

Внимание, внимание, говорит Москва, работают все радиостанци... Передаём экстренное сообщение.

Сегодня состоятся последние ходовые испытания нового супертанкера, флагмана отечественного наливного флота «Нефтяник». Впервые в мировой практике будет продемонстрирована новая разработка наших учёных -- система «Лоцман», позволяющая оптимально управлять кораблем без участия человека в акватории прямоугольной гавани. Корабль (а ведь все корабли имеют треугольную форму) умеет перемещаться по кратчайшему расстоянию из заданного начального положения в конечное между другими кораблями, стоящими на якоре в гавани. На «Нефтянике» было предусмотрено практически всё, что необходимо для безопасного испытания: и толстая стальная обшивка, и электронная система управления, и спутниковая связь с комплексом определения координат, и чуткий радар. Но в дело вмешался, как это обычно и бывает, человеческий фактор. Вовочка, сын командира корабля, тайком пробрался на секретный объект как раз перед презентацией, сел к бортовому компьютеру и решил скоротать время за любимой компьютерной игрой. В результате на компьютер проник вирус и испортил часть функций программы «Лоцман». Теперь корабль не может совершать повороты вокруг своей оси. Требуется составить программу, вычисляющую длину кратчайшего маршрута танкера к заданной точке.

\InputFile

В первой строке записаны целые числа $DX$, $DY$ -- размеры гавани ($0 < DX, DY < 10^8$). 
Строки 2-4 содержат по два целых числа -- координаты танкера (напоминаем, что корабль ``Нефтяник'', как и все остальные корабли, имеет треугольную форму). 
В строке 5 указана точка, куда должен в итоге попасть корабль (а именно, тот из его углов, координаты которого описаны во второй строке входа). 
В строке 6 записано целое число $N$ (количество других кораблей в гавани, $0 \le N \le 40$). 
Следующие $3N$ строк входа содержат координаты этих кораблей. 
Все координаты находятся в пределах гавани. Угол гавани является началом координат. Корабли не пересекаются.

\OutputFile

Выведите единственное число -- минимальную длину маршрута, округлённую до трёх знаков после десятичной точки, либо значение 
$-1$, если ``Нефтянику'' из-за повреждения навигационной программы не удастся доплыть до цели.

\Examples

\begin{example}
\exmp{
2003 2003
20 50
10 30
30 30
140 60
1
80 1000
100 20
60 20
}{
146.569
}%
\end{example}

\end{problem}
