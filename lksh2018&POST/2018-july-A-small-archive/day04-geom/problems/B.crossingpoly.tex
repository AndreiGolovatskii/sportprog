\begin{problem}{Offline-пирати}{standard input}{standard output}{1 second}{512 megabytes}

Докато целият свят се бори с online-пиратите, в морето са се върнали истинските offline-пирати! 

Вашият остров, представляващ изпъкнал многоъгълник, се е оказал обкръжен от пиратски кораби. Островът е толкова голям, че по сравнение с него корабите могат да бъдат считани за точки в равнината. 

\textbf{Сила} на групата offline-пирати ще наричаме броя двойки кораби, които се виждат помежду си. Два кораба се виждат помежду си, ако никоя точка от отсечката, която ги съединява, не лежи строго в острова. В частност, ако съединяващата ги отсечка съдържа друг кораб или преминава през границата на многоъгълника, но не съдържа негови вътрешни точки, то корабите се виждат.

Определете силата на групата от offline-пирати.


\InputFile
Първият ред съдържа две цели числа $k$ и $n$ ($3 \leq k \leq 100\,000$, $1 \leq n \leq 100\,000$)~--- броя върхове на многоъгълния остров, и броя кораби.

Следващите $k$ реда съдържат по две цели числа $px_i$, $py_i$ ($-10^9 \leq px_i, py_i \leq 10^9$)~--- координатите на върховете на многоъгълника. Върховете са дадени в ред на обхождане против часовниковата стрелка, като никои три последователни върхове не лежат на една права. Гарантира се, че многоъгълникът е изпъкнъл.

Следващите $n$ реда съдържат по две цели числа $sx_i$, $sy_i$ ($-10^9 \leq sx_i, sy_i \leq 10^9$)~--- координатите на корабите. Гарантира се, че никой от корабите не лежи нито на границата, нито във вътрешността на многоъгълника.

\OutputFile
Изведете едно число~--- силата на групата от offline-пирати.


\Example

\begin{example}
\exmp{4 5
1 1
-1 1
-1 -1
1 -1
2 1
-2 1
-2 -1
2 -1
2 1
}{7
}%
\end{example}

\Note
Разположението на корабите и на острова в теста от условието са изобразени на рисунката по-долу.

\begin{center}
\includegraphics[height=4cm]{pics/a.1}
\end{center}


\Scoring
Тестовете за тази задача са разделени на шест групи. Точките за група се присъждат само при преминаването както на всички тестове в групата, така и на всички тестове в някои от другите групи.


\medskip
\begin{center}
\begingroup
\renewcommand{\arraystretch}{1.5}
\begin{tabular}{|c|c|c|c|c|c|}
\hline
& & \multicolumn{2}{|c|}{Additional constraints} & & \\
\cline{3-4}
\raisebox{2.25ex}[0cm][0cm]{Group}
& \raisebox{2.25ex}[0cm][0cm]{Points}
& $k$ & $n$
& \raisebox{2.25ex}[0cm][0cm]{\parbox{3cm}{\centering Required groups}}
& \raisebox{2.25ex}[0cm][0cm]{Comment}
\\
\hline
0 & 0 & -- & -- & -- & Sample tests \\
\hline
1 & 16 & $k \le 10$ & $n \le 2000$ & 0 &  \\
\hline
2 & 16 & $k \le 1000$ & $n \le 2000$ & 0, 1 &  \\
\hline
3 & 16 & $k \le 200$ & $n \le 7000$ & 0, 1 &  \\
\hline
4 & 16 & $k \le 10$ & $n \le 100\,000$ & 0, 1 &  \\
\hline
5 & 36 & -- & -- & 0 -- 4 &  \\
\hline
\end{tabular}
\endgroup
\end{center}




\end{problem}

