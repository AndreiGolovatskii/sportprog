\documentclass[12pt,a4paper,oneside]{article}

\usepackage{cmap}
\usepackage[T2A]{fontenc}
\usepackage[utf8]{inputenc}
\usepackage[english,russian]{babel}
\usepackage[russian]{olymp}
\usepackage{graphicx}
\usepackage{amsmath,amssymb}
\usepackage{epigraph}
\usepackage[russian,colorlinks=true,urlcolor=red,linkcolor=blue]{hyperref}
\usepackage{enumerate}
\usepackage{datetime}
\usepackage{color}
\usepackage{lastpage}
\usepackage{import}
\usepackage{verbatim}
\DeclareGraphicsRule{.1}{mps}{*}{}  % enable *.1

\def\ID{13}
\def\NAME{День \NO{\ID}, Зачёт}%
\def\WHERE{ЛКШ.2018.Июль, Берендеевы Поляны}%
\def\DATE{23 июля 2018}%

\renewcommand{\t}{\texttt}
\renewcommand{\le}{\leqslant}
\renewcommand{\ge}{\geqslant}

\binoppenalty=10000
\relpenalty=10000
\exhyphenpenalty=10000

\newcommand{\ProblemLabel}{undefined}
\newcommand{\ProblemTL}{undefined}
\newcommand{\ProblemML}{undefined}
\newcommand{\ProblemName}{undefined}

\newcommand{\q}[1]{\langle #1 \rangle}
\newcommand\NO[1]{\t{\##1}}
\def\O{\mathcal{O}}
\def\EPS{\varepsilon}
\def\SO{\Rightarrow}
\def\EQ{\Leftrightarrow}
\def\t{\texttt}
\def\XOR{\text{ {\raisebox{-2pt}{\ensuremath{\Hat{}}}} }}
\def\LINE{\vspace*{-1em}\noindent \underline{\hbox to 1\textwidth{{ } \hfil{ } \hfil{ } }}}

\def\probl#1#2#3#4{
  \renewcommand{\ProblemName}{#1}
  \renewcommand{\ProblemLabel}{#2}
  \renewcommand{\ProblemTL}{#3}
  \renewcommand{\ProblemML}{#4}
  \input ../problems/#2.#1.tex
}
          
\newcommand{\Section}[2]{
  \hbox{\hspace{1em}}
  \vspace*{-2.5em}
  \section*{\color{#1}{#2}}
  \addcontentsline{toc}{section}{\color{#1}{#2}}
  \vspace*{-0.5em}
}

\def\myindent{\hspace*{\parindent}\unskip}

\contest
{\NAME}%
{\WHERE}%
{\DATE}%

%\sectionfont{\fontsize{8}{8}\selectfont}

\definecolor{dkgreen}{rgb}{0,0.6,0}
\definecolor{brown}{rgb}{0.5,0.5,0}

\def\compact{
  \setlength{\parskip}{0pt}
  \setlength{\itemsep}{0pt}
}

\begin{document}

\vspace*{-2.8em}

\tableofcontents

\vspace*{0.8em}
\LINE
\vspace*{0.8em}

\noindent{}В некоторых задачах большой ввод и вывод. 
Пользуйтесь \href{https://ejudge.lksh.ru/A/lib/example_io.cpp.html}{быстрым вводом-выводом}.

\vspace{0.6em}
\noindent{}В некоторых задачах нужен STL, который активно использует динамическую память (set-ы, map-ы)
\href{https://ejudge.lksh.ru/A/lib/example_mem.cpp.html}{переопределение стандартного аллокатора} ускорит вашу программу.

\pagebreak

% \Section{red}{Задачи}

\Section{red}{Задачи на 3}

% A. 0.016 : 0.1
% B. 0.022 : 0.1
% C. 0.121 : 0.5
% D. 1.287 : 3.0
% E. 0.275 : 0.7
% F. 0.154 : 0.4
% G. 0.112 : 0.5
% H. 0.159 : 0.4
% I. 0.174 : 0.5
% J. 0.051 : 0.2
% K. 0.341 : 0.7
% L. 0.003 : 0.05
% M. 1.000 : 3.0
% N. 2.267 : 5.0
% O. 0.551 : 2.0
% P. 0.092 : 0.4 

\probl{karlsson}{A}{0.1 sec}{256 mb}
\probl{evacuate}{B}{0.05 sec}{256 mb}
\probl{highways}{C}{0.5 sec}{256 mb}
\probl{firstk}{D}{3.0 sec}{256 mb}

\Section{red}{Задачи на 4}

\probl{server}{E}{0.7 sec}{256 mb}
\probl{distance}{F}{0.4 sec}{256 mb}
\probl{refrain}{G}{0.5 sec}{256 mb}
\probl{multiassignment}{H}{0.4 sec}{256 mb}

\Section{red}{Задачи на 5}

\probl{wall}{I}{0.5 sec}{256 mb}
\probl{duel}{J}{0.2 sec}{256 mb}
\probl{ligthpolygon}{K}{0.7 sec}{256 mb}
\probl{virus}{L}{0.05 sec}{256 mb}
\probl{sequence}{M}{3 sec}{256 mb}
\probl{maxsquare}{N}{5 sec}{256 mb}

\Section{red}{Гробы}

\probl{polycomp}{O}{2 sec}{256 mb}
\probl{shots}{P}{0.4 sec}{256 mb}

\end{document}
