\begin{problem}{Вирусы}{virus.in}{virus.out}{2 секунды}{64 мебибайта}{Z1}

Комитет По Исследованию Бинарных Вирусов обнаружил, что некоторые
последовательности единиц и нулей являются кодами вирусов. Комитет
изолировал набор кодов вирусов. Последовательность из единиц
и нулей называется безопасной, если никакой ее сегмент (т.е. последовательность
из соседних элементов) не является кодом вируса. Сейчас цель комитета
состоит в том, чтобы установить, существует ли бесконечная безопасная
последовательность из единиц и нулей.

\Example

Для множества кодов \texttt{\{011, 11, 0000\}} примером бесконечной безопасной
последовательности является \texttt{010101\ldots}~. Для множества \texttt{\{01, 11, 00000\}}
бесконечной безопасной последовательности не существует.

\InputFile

Первая строка входного файла \texttt{virus.in} содержит одно целое число $N$, равное
количеству всех вирусных кодов. Каждая из следующих n строк
содержит непустое слово, составленное из символов \texttt{0} и \texttt{1}~--- код вируса.
Суммарная длина всех слов не превосходит $30\,000$.

\OutputFile

Первая и единственная строка выходного файла должна содержать слово:
\begin{itemize}
\item TAK~--- если бесконечная, безопасная последовательность из нулей
и единиц сушествует;
\item NIE~--- в противном случае.
\end{itemize}

\Example

\begin{example}
\exmp{
3
01
11
00000
}{
NIE
}%
\exmp{
3
011
11
0000
}{
TAK
}%
\end{example}

\end{problem}

