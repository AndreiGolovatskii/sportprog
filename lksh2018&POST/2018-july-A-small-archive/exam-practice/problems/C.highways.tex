\begin{problem}{Платные дороги}{highways.in}{highways.out}{3 секунды}
{256 мегабайт}

Мэр одного большого города решил ввести плату за проезд по шоссе, проходящим
в районе города, чтобы снизить объем транзитного транспорта. В районе города
проходит $n$ шоссе.

Но руководство области, в которой расположен город, воспротивилось планам
мэра. Действительно --- дальнобойщики представляют собой неплохой источник
доходов для большого количества кафе и гостиниц в небольших городках.

В результате решили, что плата будет введена только на шоссе, которые 
\emph{проходят через город}. 

В городе используется развитая система метрополитена, всего в городе есть $m$ 
станций метро. Решено было, что шоссе проходит через город, если либо
одна из станций метро расположена непосредственно на шоссе, либо есть хотя бы одна
станция с каждой стороны от шоссе.

Помогите теперь мэру определить, какие шоссе проходят через город.


\InputFile
Первая строка входного файла содержит два целых числа: $n$ и $m$ ---
количество шоссе и количество станций метро, соответственно
($1 \le n, m \le 100\,000$). 

Следующие $n$ строк описывают шоссе. Каждое шоссе описывается
тремя целыми числами $a$, $b$ и $c$ и представляет собой прямую на
плоскости, задаваемую уравнением $ax+by+c=0$
($|a|,|b|,|c| \le 10^9$). 

Следующие $m$ строк входного файла описывают станции метро. Каждая станция
описывается двумя целыми числами $x$ и $y$ и представляет собой точку
на плоскости с координатами $(x, y)$ ($|x|, |y| \le 10^9$).


\OutputFile
Первая строка выходного файла должна содержать одно целое число --- количество
шоссе, которые проходят через город. Вторая строка должна содержать номера этих шоссе
в возрастающем порядке. Шоссе нумеруются от 1 до $n$ в порядке, в котором
они описаны во входном файле.


\Examples

\begin{example}
\exmp{4 2
0 1 0
1 0 1
1 1 0
1 1 -1
0 0
2 0
}{3
1 3 4 
}%
\end{example}

\end{problem}

