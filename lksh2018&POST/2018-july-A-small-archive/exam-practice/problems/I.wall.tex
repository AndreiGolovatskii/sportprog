\begin{problem}{Великая стена}{wall.in}{wall.out}{3 секунды}{64 мегабайта}

У короля Людовика двое сыновей. Они ненавидят друг друга, и король боится, что
после его смерти страна будет уничтожена страшными войнами. Поэтому Людовик
решил разделить свою страну на две части, в каждой из которых будет властвовать
один из его сыновей. Он посадил их на трон в города $A$ и $B$, и хочет построить
минимально возможное количество фрагментов стены таким образом, чтобы не
существовало пути из города $A$ в город $B$.

Страну, в которой властвует Людовик, можно упрощенно представить в виде 
прямоугольника $m\times n$. В некоторых клетках этого прямоугольника расположены
горы, по остальным же можно свободно перемещаться. Кроме этого, ландшафт в
некоторых клетках удобен для строительства стены, в остальных же строительство
невозможно.

При поездках по стране можно перемещаться из клетки в соседнюю по стороне, 
только если ни одна из этих клеток не содержит горы или построенного
фрагмента стены.

\InputFile

В первой строке входного файла содержатся числа $m$ и $n$ ($1\le m, n\le 50$).
Во второй строке заданы числа $k$ и $l$, где $0\le k, l, k+l\le mn-2$, $k$ ---
количество клеток, на которых расположены горы, а $l$ --- количество клеток,
на которых можно строить стену. Естественно, что на горах строить стену нельзя.
Следующие $k$ строк содержат координаты клеток с горами $x_i$ и $y_i$, а за ними
следуют $l$ строк, содержащие координаты клеток, на которых можно построить 
стену --- $x_j$ и $y_j$. 
Последние две строки содержат координаты городов $A$ ($x_A$ и $y_A$) и $B$ 
($x_B$ и $y_B$) соответственно.
Среди клеток, описанных в этих $k+l+2$ строках, нет
двух совпадающих. Гарантируется, что $1\le x_i, x_j, x_A, x_B\le m$ и 
$1\le y_i, y_j, y_A, y_B\le n$.

\OutputFile

В первой строке выходного файла должно быть выведено минимальное количество
фрагментов стены $F$, которые необходимо построить. В последующих $F$ строках
необходимо вывести один из возможных вариантов застройки.

Если невозможно произвести требуемую застройку, то 
необходимо вывести в выходной файл единственное число $-1$.

\Example
\parindent=0mm
\begin{example}
\exmp{%
5 5
3 8
3 2
2 4
3 4
3 1
1 3
2 3
3 3
4 3
5 3
1 4
1 5
2 1
5 5%
}{%
3
3 1
1 3
3 3}%
\end{example}

\end{problem}
