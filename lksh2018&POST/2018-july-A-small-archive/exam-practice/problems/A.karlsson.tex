\begin{problem}{Малыш и Карлсон}{karlsson.in}{karlsson.out}{1 секунда}{64 мегабайта}

На свой День рождения Малыш позвал своего лучшего друга Карлсона.
Мама испекла его любимый пирог прямоугольной формы $a \times b \times c$ сантиметров.
Карлсон знает, что у Малыша еще есть килограмм колбасы. Чтобы заполучить ее,
он предложил поиграть следующим образом: они по очереди разрезают пирог
на две ненулевые по объему прямоугольные части с целыми измерениями и съедают меньшую часть
(в случае, когда части равные, можно съесть любую).
Проигрывает тот, кто не может сделать хода (то есть когда размеры будут $1 \times 1 \times 1$).
Естественно, победителю достается колбаса.

Малыш настаивает на том, чтобы он ходил вторым.

Помогите Карлсону выяснить, сможет ли он выиграть, и если сможет --- какой должен быть
его первый ход для этого.

Считается, что Малыш всегда ходит оптимально.

\InputFile

Во входном файле содержится 3 целых числа $a$, $b$, $c$ ($1 \le a, b, c \le 5\,000$) ---
размеры пирога.

\OutputFile

В случае, если Карлсон не сможет выиграть у Малыша, выведите \texttt{NO}.
В противном случае в первой строке выведите \texttt{YES}, во второй --- размеры пирога
после первого хода Карлсона в том же порядке, что и во входном файле.

\Examples

\begin{example}%
\exmp{
1 1 1
}{
NO
}%
\exmp{
2 1 1
}{
YES
1 1 1
}%
\end{example}

\end{problem}
