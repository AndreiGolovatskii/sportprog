% Автор: Сергей Копелиович
% Источник: XXXX SPBU Championship (зима 2014-2015)

\gdef\thisproblemauthor{Сергей Копелиович}
\gdef\thisproblemdeveloper{Сергей Копелиович}
\begin{problem}{Композиция многочленов}
{polycomp.in}{polycomp.out}
{2 секунды (\textsl{3.5 секунды для Java})}{256 мебибайт}{}

Даны многочлены $f(x)$, $g(x)$, $h(x)$ над полем $\mathbb{Z}/2\mathbb{Z}$.

Найдите многочлен $f(g(x)) \bmod h(x)$.

\InputFile

Три строки ввода содержат многочлены $f$, $g$ и $h$, по одному на строке.
Каждый многочлен $p$ описывается как
$n$~$p_0$~$p_1$~$p_2$~$\dots$~$p_n$
($1 \le n \le 4000$, $p_i \in \{0, 1\}$ для всех $i$, а $p_n = 1$).
Сам многочлен $p(x)$ в таком случае равен
$p_0 + p_1 x + p_2 x^2 + \dots + p_n x^n$.
                                                                                                     
\OutputFile

Выведите ответ в том же формате.

Возможен ответ вида <<\texttt{0~0}>>, обозначающий тождественный ноль.

\Examples

\begin{example}
\exmp{%
5 0 1 0 1 0 1
2 1 1 1
4 0 1 1 0 1
}{%
1 1 1
}%
\exmp{%
2 1 1 1
3 0 0 1 1
4 1 0 1 0 1
}{%
3 1 0 0 1
}%
\end{example}

\Note

Напомним несколько определений.

Поле $\mathbb{Z}/2\mathbb{Z}$ "--- это множество из двух чисел $0$ и $1$,
в котором результаты сложения, вычитания, умножения и деления "---
это остатки по модулю $2$ от аналогичных результатов для обычных чисел.

Многочлен $f(x)$ над этим полем "--- это объект вида
$f_n \cdot x^n + f_{n - 1} \cdot x^{n - 1} + \ldots + f_1 x + f_0$,
где коэффициенты $f_n$, $\ldots$, $f_0$ "--- числа из
$\mathbb{Z}/2\mathbb{Z}$, и переменная $x$ тоже может принимать
значения из $\mathbb{Z}/2\mathbb{Z}$.
Число $n$ "--- максимальное такое, что $f_n \ne 0$ "--- называется
степенью многочлена $p(x)$.

Многочлены $a(x) = \sum \limits _k a_k x^k$ и
$b(x) = \sum \limits _k b_k x^k$ равны, если
для любого $k$ числа $a_k$ и $b_k$ равны.

Сложение и вычитание многочленов определяются покомпонентно:
$a(x) \pm b(x) = \sum \limits _k (a_k \pm b_k) \cdot x^k$.

Произведение многочленов $a(x)$ и $b(x)$ определяется как
$c(x) = \sum \limits _k c_k x^k$, где
$c_s = \sum \limits _{t = 0} ^{s} (a_t \cdot b_{s - t})$.

Многочлены можно делить друг на друга.
Если многочлен $b(x)$ не является тождественным нулём,
говорят, что $a(x) / b(x) = q(x)$ и $a(x) \bmod b(x) = r(x)$, если
$q(x) \cdot b(x) + r(x) = a(x)$, а степень $r(x)$ строго меньше
степени $b(x)$.
Несложно показать, что многочлены $q(x)$ и $r(x)$ определены однозначно.

Композиция $a(b(x))$ "--- это многочлен $\sum \limits _k a_k (b(x))^k$,
где степень многочлена определяется через произведение: $(b(x))^0 = 1$,
$(b(x))^1 = b(x)$, $(b(x))^p = b(x) \cdot (b(x))^{p - 1}$ для $p > 1$.
Коэффициенты композиции можно получить, если раскрыть скобки
и сложить коэффициенты при одинаковых степенях переменной $x$.

\end{problem}
