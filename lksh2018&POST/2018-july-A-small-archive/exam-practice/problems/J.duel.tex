\begin{problem}{Дуэль}{duel.in}{duel.out}{3 секунды}{256 мегабайт}
%\begin{problem}{Duel (Div. 2)}{duel.in}{duel.out}{3 секунды}{256 мегабайт}

%Author: Andrew Stankevich

Двое дуэлянтов решили выбрать в качестве места проведения поединка тёмную аллею.
%Alex and his rival George are preparing for the duel because of fair lady Nathalie. The duel
%will take place at a dark alley.
Вдоль этой аллеи растёт $n$ деревьев и кустов.
Расстояние между соседними объектами равно одному метру.
Дуэль решили проводить по следующим правилам.
Некоторое дерево выбирается в качестве стартовой точки.
Затем два дерева, находящихся на одинаковом расстоянии от исходного,
отмечаются как места для стрельбы.
Дуэлянты начинают движение от стартовой точки в противоположных направлениях.
Когда соперники достигают отмеченных деревьев, они разворачиваются и начинают
стрелять друг в друга.
%The alley has $n$ trees and bushes growing along, the distance between adjacent plants is one meter. Alex and George decided
%that the duel will proceed as follows. Some tree is selected as the starting point and marked accordingly.
%Two trees at equal distance from the starting tree are marked as shooting points. Alex and George will start
%at the starting tree and move in opposite directions. When they reach shooting trees they will turn around
%and shoot at each other.

Дана схема расположения деревьев вдоль аллеи.
Требуется определить количество способов выбрать
стартовую точку и места для стрельбы согласно правилам дуэли.
%Given the positions of the trees, help Alex and George find the starting tree and shooting trees. First the duelists
%would like to know the number of ways they can choose the trees.

\InputFile

Во входном файле содержится одна строка, состоящая из символов `\t{0}' и `\t{1}'
--- схема аллеи.
Деревья обозначаются символом `\t{1}', кусты --- символом `\t{0}'.
Длина строки не превосходит $100\,000$ символов.
%Input file contains a non-empty string of 0-s and 1-s that describes the alley, 0 stands for the bush (that is not
%suitable to be neither starting nor shooting point), 1 stands for the tree. The length of the string
%doesn't exceed~$100\,000$.

\OutputFile

Выведите количество способов выбрать
стартовую точку и места для стрельбы согласно правилам дуэли.
%Output the number of ways the duelists can choose starting and shooting trees.

\Examples

\begin{example}%
\exmp{
101010101
}{
4
}%
\exmp{
101001
}{
0
}%
\end{example}
\bigskip

В первом примере возможны следующие конфигурации дуэли
(стартовое дерево и деревья для стрельбы выделены жирным шрифтом):
%In the first example the following configurations of the duel are possible (starting and shooting trees are marked as bold):
{\bf 1}0{\bf 1}0{\bf 1}0101, 
10{\bf 1}0{\bf 1}0{\bf 1}01,
1010{\bf 1}0{\bf 1}0{\bf 1} и
{\bf 1}010{\bf 1}010{\bf 1}. 

\end{problem}

