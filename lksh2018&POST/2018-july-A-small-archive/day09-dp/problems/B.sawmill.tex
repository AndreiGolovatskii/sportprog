\begin{problem}{Две лесопилки}{sawmill.in}{sawmill.out}{0.1 секунд}{64 мегабайта}

От вершины до подножья холма растет $N$ старых деревьев. Районная администрация решила в
санитарных целях срубить эти деревья, а чтобы снизить стоимость мероприятия перевезти все древесину на лесопилки.
Деревья могут быть перевезены только в одном направлении -- вниз.
У подножья холма находится лесопилка, а также две дополнительные лесопилки могут быть построены на холме вдоль дороги.
Вам предстоит определить, где наиболее выгодно построить эти лесопилки, чтобы минимизировать стоимость транспортировки древесины.
Перевозка 1 килограмма древесины на 1 метр стоит 1 копейку.

\InputFile

Первая строка входного файла содержит натуральное число $N$ – количество деревьев ($1 \le N \le 20\,000$).
Деревья занумерованы от $1$ до $N$ начиная с вершины холма.
Следующие $N$ линий содержат по два целых числа $w_i$ и $d_i$ ($1 \le w_i, d_i \le 10\,000$) --
вес дерева номер $i$ и расстояние между деревьями $i$ и $i{+}1$. 
Последнее из этих чисел ($d_n$) задает расстояние от нижнего дерева до лесопилки.

\OutputFile

Выведите единственное число -- минимальную стоимость сплава деревьев вниз по реке.

\Example

\begin{example}
\exmp{
9
1 2
2 1
3 3
1 1
3 2
1 6
2 1
1 2
1 1
}{
26
}%
\end{example}

\Explanation

В примере выгодно поставить лесопилки у деревьев с номерами $3$ и $6$.

\end{problem}
