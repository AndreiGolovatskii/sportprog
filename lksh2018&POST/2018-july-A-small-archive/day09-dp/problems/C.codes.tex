\begin{problem}{Order-Preserving Codes}{codes.in}{codes.out}{2 seconds}{256 mb}

%Author: Classic, see Knuth

Binary code is a mapping of characters of some alphabet to the set of
finite length bit sequences. For example, standard ASCII code is a fixed 
length code, where each character is encoded using 8 bits.

Variable length codes are often used to compress texts taking
into account the frequencies of occurence of different characters.
Characters that occur more often get shorter codes, while characters
occuring less often --- longer ones.

To ensure unique decoding of variable length codes so called
\emph{prefix codes} are usually used. In a prefix code no code
sequence is a proper prefix of another sequence. Prefix code
can be easily decoded scanning the encoded sequence from left to
right, since no code is the prefix of another, one always knows
where the code for the current character ends and the new character
starts.

Among prefix codes, the optimal code is known, so called 
Huffman code. It provides the shortest possible length of the 
text among all prefix codes that separatly encode each 
character with an integer number of bits.

However, as many other codes, Huffman code does not preserve
character order. That is, Huffman codes for lexicographically ordered
characters are not necessarily lexicographicaly ordered.

In this problem you are asked to develop a prefix code
that would be optimal for the given text among all order-preserving 
prefix codes. Code is called
order-preserving if for any two characters the code sequence 
for the character that goes earlier in the alphabet is 
lexicographically smaller. 

Since text itself is not essential for finding the code,
only the number of occurences of each character is important,
only this data is given.

\InputFile

The first line of the input file contains $n$ --- the number
of characters in the alphabet ($2 \le n \le 2000$). The
next line contains $n$ integer numbers --- the number of 
occurences of the characters in the text for which the 
code must be developed (numbers are positive and do not
exceed $10^9$). Characters are described in the alphabetical 
order.

\OutputFile

Output $n$ bit sequences, one on a line --- the optimal
order-preserving prefix code for the described text.

\Example

\begin{example}
\exmp{
5
1 8 2 3 1
}{
00
01
10
110
111
}%
\end{example}

\end{problem}
