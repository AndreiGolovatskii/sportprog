\begin{problem}{Мостостроение}
{bridge.in}{bridge.out}
{5 секунд}{512 мебибайт}{} % Важно, что 512 !

% Автор задачи в 2009-м: Виталий Аксенов
% Автор условия в 2009-м: Павел Маврин
% Повысил в 100 раз ограничения: Сергей Копелиович

\epigraph{
  Давным давно, в 2009-м году...
}

В деревне Зайкино регулярно идут проливные дожди, в результате чего
речка Дубровка, которую обычно можно просто перешагнуть, выходит из берегов.
Чтобы можно было перейти разлившуюся реку, планируется построить
плавучий мост из брёвен, оставшихся от строительства
бани бизнесмена, поселившегося неподалёку.

Все оставшиеся брёвна имеют одинаковую толщину.
При этом есть $x$ брёвен длины $a$ и $y$ брёвен длины $b$.

Построенный мост должен состоять из $l$ рядов, каждый из которых
составлен из одного или нескольких брёвен.
Пилить брёвна нельзя, так как последняя пила утонула при разливе Дубровки.

Главный инженер хочет построить мост максимальной возможной ширины.
Ширина моста определяется по минимальной ширине ряда брёвен в нём.

Например, если нужно построить мост из семи рядов, и при этом есть
шесть брёвен длины $3$ и десять брёвен длины $2$,
то можно построить мост ширины $5$.

\begin{center}
\includegraphics{pics/bridge.1}
\end{center}

\InputFile

Ввод состоит из одного или нескольких тестовых случаев.
Каждый тестовый случай состоит из пяти целых положительных чисел
$x$, $a$, $y$, $b$ и $l$.
Каждое число не превосходит $500$. 
Общее количество брёвен в каждом тестовом случае не меньше $l$.

Обозначим $d = \max (x, a, y, b, l)$.
Гарантируется, что сумма $d$ по всем тестам не превосходит $5000$.

\OutputFile

Для каждого тестового случая на отдельной строке выведите одно число "---
максимальную возможную ширину моста.

\Example

\begin{example}
\exmp{
6 3 10 2 7
10 7 20 9 25
106 126 135 28 137
}{%
5
9
112
}%
	\end{example}

\end{problem}
